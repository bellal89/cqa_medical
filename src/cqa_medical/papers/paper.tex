\documentclass{article}
\usepackage{young4}

\begin{document}
\atitle{Извлечение именованных сущностей из тематического корпуса вопросов и ответов}{Белобородов~А.В.}
[Белобородов~А.В. Извлечение именованных сущностей из тематического корпуса вопросов и ответов]%
%% Необязательный аргумент - для включения в содержание сборника.
%% Применяется в случае наличия в заголовке принудительных переносов, сносок и т.п.
\vspace{-1.75\baselineskip}

\centerline{\it e-mail:\ bellal89@mail.ru }
\vspace{\baselineskip}
Вопросно-ответный сервис (community question answering, CQA) - это система, позволяющая обычным пользователям сети Интернет задавать вопросы в свободной форме или отвечать на них (более подробное описание дается в \cite{BELOBORODOV1}). Некоторые из таких систем имеют тематическую направленность, чтобы привлекать к общению пользователей с общими интересами (например, stackoverflow.com - CQA-ресурс, посвященный программированию). Наличие тематики позволяет извлекать из подобных сервисов информацию более качественно и детализированно, отталкиваясь от специфичных терминов.

Была поставлена задача: из набора вопросов и ответов про медицину извлечь информацию о болезнях, которыми интересуются пользователи, и о медикаментах, которые пользователи советуют для лечения. Эксперимент проводился на корпусе вопросов и ответов сервиса Ответы@Mail.Ru. Были взяты данные категорий ''Болезни, Лекарства'', ''Врачи, Клиники, Страхование'', ''Детское здоровье'', ''Отвечает врач'' за период с 01.04.2011 по 31.03.2012 - всего 128370 вопросов с ответами. Среднее количество ответов на вопрос - 5, средняя длина вопроса в словах - 7. Все слова в тексте вопросов были лемматизированы, вопросы и ответы были очищены от HTML разметки, знаков препинания.

Для идентификации терминов в тексте был составлен список из 1049 заболеваний, собранных на основе справочника фельдшера и список медикаментов из государственного реестра лекарственных средств (11926 уникальных наименований по состоянию на сентябрь 2012 года). Также был разработан метод автоматического извлечения из корпуса болезней и лекарств из составленных списков. Так как названия заболеваний и лекарств носят профессиональный характер и не являются широкоупотребительной лексикой, в них часто допускаются орфографические ошибки. Поэтому на основе работы \cite{BELOBORODOV2} был реализован нечеткий поиск с помощью индекса 3-грамм и последующего вычисления расстояния Левенштейна в зависимости от длины слова. Например, для слова диабет с частотой 723 были найдены слова диабед (16), диобет (5), деабет (3), диабетя (1), диает (1), диабеа (1), 7диабет (1) - еще 4% от точных совпадений.
Из всех вопросов выбирались удовлетворяющие следующим условиям:
1. В тексте вопроса упоминается заболевание из составленного списка;
2. В тексте ответов на вопрос упоминается медикамент из составленного списка.
Тем самым моделировалась ситуация, когда от спрашивающего пользователя поступает жалоба на заболевание, а от отвечающих пользователей - совет использовать определенное лекарство. Всего найдено 24972 таких вопроса (19%). Далее пары (заболевание, лекарство) упорядочивались по количеству вопросов, в которых были встречены. Очевидно, что если пара встретилась в небольшом количестве вопросов, о ней нельзя сказать ничего определенного - это может быть несколько случайных упоминаний не по теме. Напротив, наиболее частые пары выражают мнение людей о конкретном лекарстве относительно конкретной болезни.

\begin{table}[h]
\caption{Наиболее часто упоминаемые медикаменты для некоторых заболеваний}
\begin{center}
\begin{tabular}{|c|c|c|c|}
\hline
Ангина & Молочница & Рана & Герпес \\
\hline
Люголь (134) & Флюкостат (183) & Перекись (149) & Ацикловир (286) \\
\hline
Ромашка (125) & Флуконазол (126) & Зеленка (96) & Зовиракс (136) \\
\hline
Йод (123) & Кандид (115) & Йод (88) & Сера (92) \\
Фурацилин (105) & Нистатин (112) & Левомеколь (73) & Корвалол (51) \\
\hline
Шалфей (86) & Клотримазол (109) & Спирт (51) & Фенистил (34) \\
\hline
\end{tabular}
\end{center}
\end{table}

В будущих исследованиях планируется оценить точность извлекаемой информации о заболеваниях и медикаментах с помощью международной классификации болезней МКБ-10, которая находится в открытом доступе в сети Интернет \footnote{Регистр лекарственных средств России --- http://www.rlsnet.ru/mkb_tree.htm}%. В МКБ-10 для большинства заболеваний представлен список препаратов, рекомендуемых для лечения. Если точность извлекаемых данных окажется на достаточно высоком уровне, то рассмотренный метод можно будет использовать для автоматического извлечения тематических данных из содержимого вопросно-ответных сервисов.

\begin{thebibliography}{3}

%% Пробелы в метках не допускаются!

\bibitem{BELOBORODOV1}
{\it Белобородов~А.} Сравнение трех мер семантической близости вопросов в социальных вопросно-ответных сервисах // Труды Международной (43-й Всероссийской) молодежной школы-конференции ''Современные проблемы математики''. 2012.

\bibitem{BELOBORODOV2}
{\it Norvig~P.} How to Write a Spelling Corrector // Online; visited February 22, 2008. http://norvig.com/spell-correct.html.

\end{thebibliography}

\end{document}

\hline